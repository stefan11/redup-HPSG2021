%%%%%%%%%%%%%%%%%%%%%%%%%%%%%%%%%%%%%%%%%%%%%%%%%%%%%%%%%
%%   $RCSfile: example.tex,v $
%%  $Revision: 1.1 $
%%      $Date: 2004/08/11 09:52:43 $
%%     Author: Stefan Mueller (CL Uni Bremen)
%%    Purpose: Example file for contributions to the HPSG proceedings
%%   Language: LaTeX
%%%%%%%%%%%%%%%%%%%%%%%%%%%%%%%%%%%%%%%%%%%%%%%%%%%%%%%%%

% use this to check the margins
\documentclass[11pt,a4paper,fleqn,draft]{article}

% use the following once there are no problems with the margins
%\documentclass[11pt,a4paper,fleqn]{article}

\usepackage{ifxetex}
\ifxetex
% OTF Font loading
\usepackage{unicode-math}
\usepackage{fontspec}
\usepackage{xunicode} % For tipa macros etc.  Must be loaded before
                      % font selection, yet after unicode-math.sty

\usepackage{article-ex}

%\setmainfont{Times New Roman} 

\setmainfont[
   Renderer=ICU
   ,Path=./fonts/,
   ,Ligatures={TeX,Common}
   ,BoldFont = FreeSerifBold_B.otf
   ,ItalicFont = FreeSerifItalic_B.otf
   ,BoldItalicFont = FreeSerifBoldItalic_B.otf
   ,BoldSlantedFont = FreeSerifBold_B.otf
   ,SlantedFont    = FreeSerif_B.otf
   ,SlantedFeatures = {FakeSlant=0.25}
   ,BoldSlantedFeatures = {FakeSlant=0.25}
   ]{FreeSerif_B.otf}

% \newfontfamily\tgtermes{TeX Gyre Termes}
% \makeatletter
%   \begingroup
%     \tgtermes
%     \DeclareFontShape{\f@encoding}{\rmdefault}{m}{sc}{%
%       <-> ssub * \f@family/m/sc}{}
%     \DeclareFontShape{\f@encoding}{\rmdefault}{bx}{sc}{%
%       <-> ssub * \f@family/bx/sc}{}
%   \endgroup
% \makeatother


\setsansfont[
    Scale=MatchLowercase
   ,Renderer=ICU
   ,Ligatures={TeX,Common}
   ,Path=./fonts/,
   ,BoldFont = FreeSansBold_B.otf
   ,ItalicFont = FreeSansOblique_B.otf
   ,BoldItalicFont = FreeSansBoldOblique_B.otf
   ]{FreeSans_B.otf}


%\setmonofont[Path=./fonts/]{FreeMono_B.otf}

%\setmathfont{TeX Gyre Termes Math}

\else
\usepackage[T1]{fontenc}
\usepackage[utf8]{inputenc}

\fi


\usepackage{microtype}
\LoadMicrotypeFile{ptm}

% End OTF font loading 


\pagestyle{empty}

% tree
%\usepackage{forest}
% Needed for the "actual forest version"
%\useforestlibrary{linguistics}
%\forestapplylibrarydefaults{linguistics}

\usepackage{langsci-forest-setup}

\usepackage{unified-biblatex}
\newcommand{\citegen}[2][]{\citeauthor{#2}'s (\citeyear*[#1]{#2})}

\ExecuteBibliographyOptions{doi=false,url=false}
\AtBeginBibliography{\small}

\bibliography{Bib}

%\usepackage[sectionbib]{natbib}
%\setlength{\bibsep}{0pt plus 0.3ex}
%\setcitestyle{citesep={;}}

\usepackage{etoolbox}
\newrobustcmd{\disambiguate}[3]{#2~#3}

%% Use this to add space in glossings e.g. for a "[" in the original you write \hapsceThis{[}
%% hspace over width of something without showing it
\newlength{\LSPTmp}
\newcommand*{\hspaceThis}[1]{\settowidth{\LSPTmp}{#1}\hspace*{\LSPTmp}}

% german.sty has nice hyphenation support
\usepackage{german}
\selectlanguage{USenglish}
%hyphenation
\usepackage{hyphenat}

% this package supports scaling for instance if your AVMs do not fit the page
\usepackage{graphicx}
%\resizebox{\linewidth}{!}{%
%\ms{
%     synsem & \onems{ loc$|$cat$|$subcat {\rm del(\ibox{2},\ibox{1})}\\
%                     }\\
%      head-dtr & \onems{ synsem$|$loc$|$cat$|$subcat \ibox{1} \\
%                      }\\
%      non-head-dtrs & \sliste{\onems{ synsem \ibox{2}  \\ }}\\[2mm]
%}}

% includes page numbers for references
%\usepackage{varioref}

% If you want to make your paper more usable for readers,
% insert the following code.
% It links references to sections, examples, and citations
% in the document and provides bookmarks for sections and subsections.

% For URLs load the style file url.sty , You may then type \url{http://hpsg.fu-berlin.de/~stefan/}
% without any hassle regarding the `~'. 
%\usepackage[hyphens]{url}
%\urlstyle{same}

\usepackage[bookmarks=true,bookmarksopen=true,hyperindex=true,breaklinks=true,draft=false,plainpages=false,hyperfootnotes=false,%
colorlinks=false, pdfborder={0 0 0}%
%,pdftex=true  % use this if you use pdflatex
%,ps2pdf=true  % use this if you use ps2pdf, i.e. if you use tree-dvips
]{hyperref}
\hypersetup{colorlinks=false, pdfborder={0 0 0}}


%% Symbols
\usepackage{amsmath}

% figures
\usepackage{float}
\usepackage{subcaption}
\usepackage{tikz}
\usetikzlibrary{patterns, matrix}
\usepackage{multicol}



% avm
\usepackage{langsci-avm}

% examples and glossing
\usepackage{langsci-gb4e}
\usepackage{jambox}

\usepackage{xspace}

\newlength{\MyetagLength}
\settowidth{\MyetagLength}{{$\scriptstyle 1$}}

%% indexed box
\newcommand{\iboxt}[1]{{%
  \setlength{\fboxsep}{1.25pt}%
  \fbox{$\scriptstyle #1$}%
}}


\newcommand{\ibox}[1]{%
%  \iboxt{#1}\,%       why this extra space???? 20.02.2018
\iboxt{#1}%
}

\newcommand{\iboxb}[1]{(\,\iboxt{#1}\,)}

%% empty box
\newcommand{\etag}{\ibox{\rule{0ex}{1.1ex}\hspace{\MyetagLength}}\xspace}

%% type
\newcommand{\type}[1]{{\normalfont\itshape #1\/}}

%% list of phons
\newcommand{\phonliste}[1]{%
\mbox{%
$%
%
\left\langle \mbox{\normalfont\itshape#1} \right\rangle%
$%
%\\[-1.5mm]
}%
}

\newcommand{\sliste}[1]{%
\mbox{%
$\left\langle\mbox{\upshape #1}\right\rangle$}%
}

\newcommand{\impl}{$\Rightarrow$\xspace}


\usepackage{xcolor}
\newcommand{\added}[1]{\textcolor{red}{#1}}

\newcommand{\changed}[1]{\textcolor{orange}{#1}}
\newcommand{\deleted}[1]{\textcolor{blue}{#1}}



\let\textbf\emph


\begin{document}
% Title is generated automatically.


\begin{abstract}

The current study presents an HPSG analysis for verbal reduplication in Mandarin Chinese. 
After discussing its interaction with \emph{Aktionsarten} and aspect markers, 
we argue that it is a morphological rather than syntactic process.
We put forward a lexical rule for verbal reduplication in Mandarin Chinese
and the different forms of reduplication are captured in an inheritance hierarchy.
The interaction between verbal reduplication and aspect marking is handled by multiple inheritance.
This analysis covers all forms of verbal reduplication in Mandarin Chinese 
and has none of the shortcomings of the previous analyses.

\end{abstract}

%\setcounter{footnote}{2}
%\renewcommand{\thefootnote}{\fnsymbol{footnote}}
%\footnotetext{
%I thank X and Y.}
%\renewcommand{\thefootnote}{\arabic{footnote}}
%\setcounter{footnote}{0}


\section{Introduction}\label{ch:intro}

In Mandarin Chinese, verbs can be reduplicated to express a delimitative aspectual meaning \citep[e.g.][]{Chao1968, Chen2001, Dai1997, Li1996, LiThompson1981, Tsao2001, XiaoMcEnery2004, Yang2003, Zhu1998}. 
It means that the event or state denoted by the verb happens in a short duration and/or a low
frequency \citep[155]{XiaoMcEnery2004}, such as illustrated in (\ref{ex:redup-ex}).\footnote{In the
  following paper, reduplications in the example sentences will be set in italics.}
Thus, verbal reduplication in Mandarin Chinese is often translated as doing something ``a little bit/for a little while”.

\setlength{\columnsep}{-27pt}
\ea\label{ex:redup-ex}
\begin{multicols}{2}
	\ea
	\gll qing ni chang zhe dao cai.\\
	please you taste this \textsc{clf} dish\\
	\glt `Please taste this dish.'
	
	\ex
	\gll qing ni \textbf{chang}-\textbf{chang} zhe dao cai.\\
	please you taste-taste this \textsc{clf} dish\\
	\glt `Please taste this dish a little bit.' 
	\z
\end{multicols}
\z

The current study tries to determine a suitable formal and unified analysis for the structure of verbal reduplication in Mandarin Chinese.
It provides a novel HPSG analysis to this phenomenon and avoids the problems of previous approaches.

After this introduction, we will present in Section \ref{ch:phen} the forms and syntactic distribution as well as the semantics of verbal reduplication in Mandarin Chinese. 
Importantly, we restrict the object of this study to the AA, A-\emph{yi}-A, A-\emph{le}-A, A-\emph{le}-\emph{yi}-A, ABAB and AB-\emph{le}-AB forms of verbal reduplication in Mandarin Chinese.
We will also discuss in this section the question of whether the reduplication is a morphological or syntactic process with the help of corpus data.
In Section \ref{ch:prev}, we will discuss the advantages and drawbacks of previous approaches. 
We will present a new HPSG account for verbal reduplication in Mandarin Chinese in Section \ref{ch:HPSG-redup}.

In addition to introspection, the Modern Chinese subcorpus of the corpus of the \emph{Center for Chinese Linguistics of Peking University} (CCL) \citep{Zhanetal2003, Zhanetal2019} was also consulted. 
Other examples from novels and plays written by native speakers were also considered.


%%%%%%%%%%%
%%%%%%%%%%% phen
\section{The phenomenon}\label{ch:phen}

This section introduces the fundamental grammatical behaviors of verbal reduplication in Mandarin Chinese. 
After illustrating its forms, syntactic distribution and semantics, 
we discuss its interaction with \emph{Aktionsarten} and other aspect markers in Mandarin Chinese, 
and the question of whether it is better analyzed as a morphological or a syntactic phenomenon.



Verbal reduplication in Mandarin Chinese takes the forms listed  in (\ref{ex:redup-forms}).

\footnotetext[2]{This form is more common in Taiwan than in Mainland China.}
\ea\label{ex:redup-forms}
	\ea for monosyllabic verbs: \emph{shuo} `say'
		\ea \gll shuo-shuo\\
		say-say\\\jambox{AA}
		\ex \gll shuo-yi-shuo\\
		say-one-say\\\jambox{A-\emph{yi}-A}
		\ex \gll shuo-le-shuo\\
		say-\textsc{pfv}-say\\\jambox{A-\emph{le}-A}
		\ex \gll shuo-le-yi-shuo\\
		say-\textsc{pfv}-one-say\\\jambox{A-\emph{le}-\emph{yi}-A}
		\ex \gll shuo-shuo-kan\\
		say-say-look\\\jambox{AA-\emph{kan}}
		\ex \gll shuo-kan-kan\\
		 say-look-look\\\jambox{A-\emph{kan}-\emph{kan}\footnotemark}
		\z
	\ex for disyllabic verbs: \emph{lai-wang} come-go `come and go/communicate'\\
		\ea \gll lai-wang-lai-wang\\
		come-go-come-go\\ \jambox{ABAB}
		\ex \gll lai-wang-le-lai-wang\\
		come-go-\textsc{pfv}-come-go\\ \jambox{AB-\emph{le}-AB}
		\ex \gll lai-lai-wang-wang\\
		come-come-go-go\\ \jambox{AABB}
		\z
	\ex\label{ex:forms-VO} for V-O compounds: \emph{chang-ge} sing-song `sing'\\
		\ea \gll chang-chang-ge\\
		sing-sing-song\\ \jambox{AAB}
		\ex \gll chang-yi-chang-ge\\
		sing-one-sing-song\\ \jambox{A-\emph{yi}-AB}
		\ex \gll chang-le-chang-ge\\
		sing-\textsc{pfv}-sing-song\\ \jambox{A-\emph{le}-AB}
		\z
	\z
\z



\citet{Arcodiaetal2014}, \citet{Fan1964}, \citet{MelloniBasciano2018} and \citet{Xie2020} compared the AA, ABAB and AABB forms of reduplication 
and found a number of differences of the AA, ABAB forms compared with the AABB form in terms of their semantics, productivity, syntactic distribution and origin. 
The current study will only focus on the AA, A-\emph{yi}-A, A-\emph{le}-A, A-\emph{le}-\emph{yi}-A, ABAB and AB-\emph{le}-AB forms of verbal reduplication in Mandarin Chinese.\footnote{For sake of simplicity, 
the term \emph{reduplication} will be used in the following text to refer specifically to the AA, A-\emph{yi}-A, A-\emph{le}-A, A-\emph{le}-\emph{yi}-A, ABAB and AB-\emph{le}-AB forms of verbal reduplication in Mandarin Chinese, if not specified otherwise.}
AA-\emph{kan}, A-\emph{kan}-\emph{kan}, AAB, A-\emph{yi}-AB, A-\emph{le}-AB will also be mentioned occasionally to provide further arguments.




The reduplication has a similar syntactic distribution as an unreduplicated verb. 
The reduplication cannot be aspect marked, though, except with the perfective aspect marker \emph{le} (for further discussions see Section \ref{sec:aspM}). 
The reduplication is incompatible with an expression that quantifies the duration or the extent of the event expressed in the sentence, as in (\ref{ex:redup-quan}) \citetext{\citealp[114--115]{Chen2005}; \citealp[83--84]{Li1998}}.
This is probably because the reduplication already contains a quantity meaning \citetext{\citealp[114--115]{Chen2005}; \citealp[84]{Li1998}}, namely a short duration or a small extent.

\settowidth\jamwidth{[Test jambox]}

\ea \label{ex:redup-quan}
\begin{multicols}{2}
		\ea[]{\gll ta yi tian pao shi li.\\
		he one day run ten mile\\
		\glt `He runs ten miles a day.'}
		\ex[*]{\gll ta yi tian \textbf{pao}-\textbf{pao} shi li.\\
		he one day run-run ten mile\\}
		\z
\end{multicols}
	\z



The reduplication has a \emph{delimitativeness} meaning \citep[e.g.][]{Chao1968, Chen2001, Dai1997, Li1996, LiThompson1981, Tsao2001, XiaoMcEnery2004, Yang2003, Zhu1998}. 
The semantics of the reduplication has the properties of transitoriness,  holisticity  and dynamicity \citetext{\citealp[70--79]{Dai1997}; \citealp[155--159]{XiaoMcEnery2004}}.
It presents the situation as a  transitory and non\hyp{}decomposable whole,
which involves not only changes in the initiation and termination of an event, but also changes in the transitory process itself.
Compared to (\ref{ex:le-dyn-look}), which could mean that the protagonist kept staring at the footprint,
(\ref{ex:redup-dyn-look}) indicates that the protagonist took a brief look or several brief looks at the footprint and looked away in the end, which is a process full of changes.

\ea
  \ea\label{ex:le-dyn-look}
    \gll Wu Xumang kan-le zuo-an shi liuxia de jiaoyin \ldots\\
    Wu Xumang look-\textsc{pfv} commit-crime when leave \textsc{de} footprint\\ \jambox{\citep[158]{XiaoMcEnery2004}}
    \glt `Wu Xumang looked at the footprint left when the crime was committed.'
  \ex\label{ex:redup-dyn-look}
    \gll Wu Xumang \textbf{kan-le-kan} zuo-an shi liuxia de jiaoyin \ldots\\
    Wu Xumang look-\textsc{pfv}-look commit-crime when leave \textsc{de} footprint\\ \jambox{\citep[158]{XiaoMcEnery2004}}
    \glt `Wu Xumang looked a little bit at the footprint left when the crime was committed.'
  \z
\z

As for the other forms of the reduplication, 
A-\emph{yi}-A is considered to have the same core semantics as AA,
despite being described to tend to have different pragmatic uses \citep{Yang2003}. 
The semantics of A-\emph{le}-A can be deduced compositionally from its structure. 
It is a combination of the perfective aspect and delimitativeness, ``conveying a transitory event which has been actualized'' \citep[151]{XiaoMcEnery2004}.
AA-\emph{kan} and A-\emph{kan}-\emph{kan} are described to express a ``try \ldots{} and find out'' meaning \citep[63]{Cheng2012}.
\citet[290]{Tsao2001} also observed that the tentative meaning is particularly prominent when the reduplication is followed by \emph{kan} `look'.
We consider the tentativeness implied by these two forms to be a pragmatic extension of delimitativeness.
The tentative meaning is made prominent by the verb \emph{kan} `look',
and the whole structure can be understood as ``do A a little bit and see''.




%%%%%%%%
\subsection{Interaction with \emph{Aktionsarten}}\label{sec:Aktionsarten}

Previous research often claimed that the reduplication can only be used for verb classes of certain \emph{Aktionsarten}, while it is infelicitous for other ones.
\citet[277--278]{Hong1999} and \citet[234--235]{LiThompson1981} suggested that reduplication is only possible for volitional activity verbs.
\citet[70--71]{Dai1997} and \citet[290]{Tsao2001} both considered that reduplication can only be used in dynamic situations.
The former further claimed that achievement verbs cannot be reduplicated.
\citet[20]{Arcodiaetal2014}, \citet{BascianoMelloni2017} and \citet[155]{XiaoMcEnery2004} proposed that only [$+$dynamic] and [$-$result] verbs can be reduplicated.
This means that the reduplication can only interact with activities and semelfactives, but not with states and achievements.

\citet[53]{Chen2001} and \citet[10--11]{Yang2003} acknowledged that the reduplication of non\hyp{}volitional verbs is more restricted than that of volitional ones.
But \citet[381--382]{Zhu1998} listed a number of non\hyp{}volitional predicates that can be reduplicated.
We found the examples shown in (\ref{ex:nonvol}) in CCL where non\hyp{}volitional verbs \emph{weiqu} `feel wronged', \emph{ren-xing} `be willful' and \emph{diao} `drop' are reduplicated.

\ea\label{ex:nonvol}
\ea
\gll dajia ye zhihao \textbf{weiqu-weiqu} le.\\
everybody also can.only feel.wronged-feel.wronged \textsc{ptc}\\ \jambox{(CCL)}
\glt `Everybody can only feel wronged a little bit.'

\ex
\gll ta-men neng zuo de buguo shi \textbf{ren-ren-xing} shua dian'er xiao piqi \textbf{diao-diao} yanlei shenme de.\\
she-\textsc{pl} can do \textsc{de} just be be.willful-be.willful-temperament play a.little small temper drop-drop tear what \textsc{de}\\ \jambox{(CCL)}
\glt `What they can do is just to be a little bit willful, to lose their temper a little bit and to drop a little bit of tears or something.'
\z
\z

It is true that  the reduplication of stative and achievement verbs is not as easily acceptable as that of activities and semelfactives.
Compared to the questionable reduplication of the stative verb \emph{bing} `be sick' in (\ref{ex:redup-stat})
and that of the achievement verb \emph{ying} `win' in (\ref{ex:redup-achiV}), 
the reduplication of the activity verb \emph{kan} `watch' in (\ref{ex:redup-actV})
and that of the semelfactive verb \emph{kesou} `cough' in (\ref{ex:redup-semel}) is readily acceptable.

\ea
\ea[?]
{\gll ta \textbf{bing-bing} jiu hao le.\\
he be.sick-be.sick then well \textsc{ptc}\\ \jambox{\citep[155]{XiaoMcEnery2004}}
Intended: `He was sick for a little while and then got well.'}\label{ex:redup-stat}

\ex[?]
{\gll ta \textbf{ying}-\textbf{le}-\textbf{ying} na chang bisai.\\
he win-\textsc{pfv}-win that \textsc{clf} competition\\ \jambox{\citep[155]{XiaoMcEnery2004}}
\glt Intended: `He won that competition a little bit.'}\label{ex:redup-achiV}

\ex[]
{\gll ta \textbf{kan}-\textbf{le}-\textbf{kan} na chang bisai.\\
he watch-\textsc{pfv}-watch that \textsc{clf} competition\\
\glt `He watched that competition for a little while.' \\}\label{ex:redup-actV}

\ex[]
{\gll ta \textbf{kesou-kesou} jiu hao le.\\
he cough-cough then well \textsc{ptc}\\
\glt `He coughed a little bit and then got well.'}\label{ex:redup-semel}
\z
\z

However, examples such as those in (\ref{ex:redup-achi-stat1}) -- (\ref{ex:redup-achi-stat2}) were found in novels and plays written by native speakers and  sentences like (\ref{ex:redup-achi-stat4}) and (\ref{ex:redup-achi-stat5}) were constructed by native speaker linguists.
Here, achievement verbs like \emph{wang} `forget' and \emph{sheng} `give birth to' and stative verbs like \emph{shutan} `be comfortable' and \emph{bing} `be sick' are reduplicated.

\footnotetext[4]{Liu, Zhen. 1963. \emph{Chang chang de liushui [Long long water]}, 72. Beijing: The Writers Publishing House.}
\footnotetext[5]{Tian, Han. 1959. \emph{Tianhan xuanji [Selected works of Tianhan]}, 122. Beijing: People's Literature Publishing House.}
\ea
\ea\label{ex:redup-achi-stat1}
\gll deng ren-men ba zhe jian shi \textbf{wang}-\textbf{wang} zai shuo ba.\footnotemark\\
wait people-\textsc{pl} \textsc{ba} this \textsc{clf} incident forget-forget then talk \textsc{ptc}\\
\glt `Let's wait until people forget this incident a little bit and then talk about it.'\\

\ex\label{ex:redup-achi-stat2}
\gll huitou mo ge zao \textbf{shutan}-\textbf{shutan} ba.\footnotemark\\
later wipe \textsc{clf} bath be.comfortable-be.comfortable \textsc{ptc}\\
\glt `Let's take a bath later and be comfortable for a little while.'\\

\ex\label{ex:redup-achi-stat4}
\gll wo zhen xiang \textbf{bing-yi-bing}, xie ta ge shi tian ban yue.\\
I really want be.sick-one-be.sick rest it \textsc{clf} ten day half month\\ \jambox{\citep[54]{Chen2001}}
\glt `I really want to be sick for a little while and rest for ten days or half a month.'

\ex\label{ex:redup-achi-stat5}
\gll jiao ta \textbf{sheng-sheng} xiaohai, jiu zhidao zuo muqin de gan-ku le.\\
let she give.birth.to-give.birth.to child then know \textsc{cop} mother \textsc{de} sweet-bitter \textsc{ptc}\\ \jambox{\citep[112]{Chen2005}}
\glt `Let her try to give birth to a child and then she will know the bittersweetness of being a mother.'
\z
\z



This shows that although the reduplication does have a tendency to interact with volitional verbs and with activities and semelfactives due to its dynamic meaning, 
this is by no means a rigid constraint, 
and non\hyp{}volitional verbs, states and achievements can be reduplicated in certain contexts as well.




%%%%%%%%
\subsection{Interaction with aspect markers}\label{sec:aspM}

As mentioned above, the reduplication can only be marked by the perfective aspect marker \emph{le} but not other aspect markers.\footnote{
There is no consensus on which elements exactly are considered aspect markers in Mandarin Chinese. We only discuss the most commonly recognized ones here.
}
We consider this incompatibility be due to semantic reasons.

\citet[Ch. 4]{XiaoMcEnery2004} considered \emph{le}, \emph{guo} and reduplication to indicate perfective aspects.
The perfective aspect marker \emph{le} is compatible with reduplication while the experiential aspect marker \emph{guo} is not.
\emph{Le} ``can focus on both heterogeneous internal structures and changing points'' \citep[129]{XiaoMcEnery2004}.
It is compatible with the reduplication, because its dynamicity can relate to not only the termination or instantiation of an event (a point of change), but also the process of the situation, just like that of the reduplication.

On the other hand, the experiential aspect marker \emph{guo} cannot co\hyp{}occur with a reduplicated verb, 
because its dynamicity attributes to an ``experiential change'' \citep[148]{XiaoMcEnery2004}, 
namely that a situation has been experienced historically and that ``the final state of the situation no longer obtains'' at the reference time \citep[144]{XiaoMcEnery2004}. 
It is clear that \emph{guo} only indicates a change at the termination of a situation and cannot express the dynamicity within a situation.
Hence, it is incompatible with the semantics of the reduplication.

Due to the holistic semantics of the reduplication, it is incompatible with imperfective aspect markers: the durative aspect marker \emph{zhe} and the progressive aspect marker \emph{zai}, as both only focus on  a part of the situation and do not view the situation as a whole \citep[Ch. 5]{XiaoMcEnery2004}.

From the illustration above, it seems that due to its semantics, reduplication can only be marked by \emph{le} but not the other aspect markers.







%%%%%%%%%%%%
\subsection{Word vs. phrase}\label{sec:word}

The literature on reduplication makes different assumptions on whether it is a morphological or syntactic phenomenon.
\citet{Chao1968} and \citet{LiThompson1981} listed reduplication under morphological processes. 
\citet{Arcodiaetal2014}, \citet{BascianoMelloni2017}, \citet{MelloniBasciano2018}, \citet{Xie2020}, \citet{Xiong2016}, \citet{YangWei2017}, on the other hand, claimed it to be syntactic.
This section reviews the arguments in \citet{Xie2020}, applies the tests proposed by \citet{Duanmu1998} and \citet{Schaefer2009} to distinguish words from phrases in Mandarin Chinese. 
The results argue for a morphological status of reduplication.


\citet{Xie2020} compared the AA and the ABAB forms of reduplication with the AABB form and claimed that AA and ABAB are syntactic processes while AABB is morphological.
She pointed out that AA and ABAB behave differently from AABB in their productivity, possibility of \emph{le} insertion, categorial stability, transitivity, and input/output constraints.
While AA and ABAB are highly productive, AABB shows low productivity. 
\emph{Le} can be inserted freely into AA and ABAB but not into AABB.
The output of AA and ABAB does not change the grammatical category of the input (verb), but the output of AABB could have other categories such as adverb or adjective.
AA and ABAB do not change the valency of the input verb, but AABB makes a transitive verb intransitive. 
The two groups also have different input and output constraints. 
\citet{Xie2020} claimed that only dynamic and volitional verbs can undergo AA or ABAB reduplication (but see Section \ref{sec:Aktionsarten}).
On the other hand, AABB requires its input to be a complex verb whose constituents are either synonymous, antonymous or logically coordinated. 
Moreover, the output of AABB has an increasing meaning, i.e. an event happens repeatedly or continuously, as opposed to the delimitative meaning of AA and ABAB.

However, a morphological process can be productive, and it does not necessarily change the category or valency of the input.
Further, if \emph{le} is considered to be a morphological element \citep[e.g.][]{Huangetal2009, MuellerLipenkova2013}, the insertion of \emph{le} does not have to be viewed as a syntactic
process either.
It seems that \citet{Xie2020} only showed that AA and ABAB are different processes than AABB, but not necessarily that the former is syntactic while the latter morphological.

It is, therefore, necessary to resort to other tests that are intended to distinguish words from phrases. 
\citet{Duanmu1998} and \citet{Schaefer2009} proposed the following four tests to
distinguish words from phrases in Mandarin Chinese: semantic compositionality, phrasal extension,
phrasal substitution and conjunction reduction.\footnote{It is important to note that none of these criteria are sufficient or necessary to determine the word or phrase status of an expression. Nevertheless, they together might suggest which of the two statuses is more likely.}

The semantic criterion is that the meaning of a phrase is usually built up in a compositional way
while that of a word is usually not \citetext{\citealp[140]{Duanmu1998}; \citealp[275]{Schaefer2009}}. 
The meaning of the reduplication is not compositional, 
as it does not mean that the event denoted by the verb happens twice or multiple times, 
but rather that the event happens for a short duration and/or a low frequency.


The first syntactic test is phrasal extension, namely the addition of optional elements \citetext{\citealp[150]{Duanmu1998}; \citealp[280]{Schaefer2009}}. 
Optional elements that can possibly appear in a phrase should be able to be added into it.
Subparts of a phrase should be able to be modified separately. 
And these should not be possible for a word.
As illustrated in (\ref{ex:redup-forms}) in Section~\ref{ch:intro}, the reduplication can only be separated by \emph{le} and \emph{yi}. 
As mentioned above, whether aspect markers are considered to be morphological or syntactic elements depends on the theoretical framework (and possibly the target language).
And the status of \emph{yi} is unclear.
Also, each element in the reduplication cannot be modified by itself. 
Compared to (\ref{ex:ext-redup1}), where the adverbial \emph{qingsheng de} `quietly' modifies the whole reduplication, (\ref{ex:ext-redup2}) is ungrammatical, 
as the adverbial cannot modify the second element in the reduplication alone.

\setlength{\columnsep}{-10pt}
\ea
\begin{multicols}{2}
  \ea[]{\label{ex:ext-redup1}
  \gll ta qingsheng de \textbf{xiao-le-xiao}.\\
  he quietly \textsc{de} laugh-\textsc{le}-laugh\\
  \glt `He quietly laughed a little bit.'}
  
  \ex[*]{\label{ex:ext-redup2}
  \gll ta \textbf{xiao}-\textbf{le} qingsheng de \textbf{xiao}.\\
  he laugh-\textsc{pfv} quietly \textsc{de} laugh\\}
  \z
\end{multicols}
\z



The second syntactic test is phrasal substitution, namely the substitution of smaller exemplars of a specific category with a full blown XP \citetext{\citealp[152]{Duanmu1998}; \citealp[280]{Schaefer2009}}. 
If a part of an expression is actually an XP that only contains one element, a full realization of this XP should be possible as well.
Otherwise, this expression is considered to be a word.
In a reduplication structure, it is ungrammatical to substitute each element with a full VP (\ref{ex:sub-redup}).

\ea\label{ex:sub-redup}
\begin{multicols}{2}
  \ea[]{\label{ex-he-tasted-the-soup-a-little-bit}
  \gll ta \textbf{chang}-\textbf{le}-\textbf{chang} tang.\\
    he taste-\textsc{pfv}-taste soup\\
    \glt `He tasted the soup a little bit.'}
    
    \ex[*]{
    \gll ta \textbf{chang} \textbf{tang} \textbf{le} \textbf{chang} \textbf{tang}.\\
    he taste soup \textsc{pfv} taste soup\\}
  \z
\end{multicols}
\z


The third syntactic criterion is conjunction reduction. It should only be possible for coordinated phrases 
and not for coordinated words \citetext{\citealp[137]{Duanmu1998}; \citealp[283]{Schaefer2009}}. 
For the reduplication, conjunction reduction does not seem to be possible. 
In (\ref{ex:co-redup1}), the reduplication \emph{jiao-jiao} `chew a little bit' is coordinated with a simple verb \emph{mo} `apply' together with the adverbial \emph{yidian} `a little bit'.
Without the adverbial \emph{yidian} `a little bit', \emph{mo} `apply' by itself cannot express the additional `a little bit' meaning even when it is coordinated with a reduplicated verb.
Similarly, in (\ref{ex:co-redup2}), the reduplication \emph{kan-le-kan} `look a little bit' is coordinated with the predicate \emph{zou-le chulai} `walked out'. The verb in the latter case is not reduplicated and it cannot express the delimitative meaning either.

\ea
  \ea\label{ex:co-redup1}
  \gll wujian gong-xiu mo dian bohe-gao huo \textbf{jiao}-\textbf{jiao} kouxiangtang.\\
    midday work-break apply a.little mint-cream or chew-chew chewing.gum\\ \jambox{(CCL)}
    \glt `During the working break in the midday, apply a little bit of mint cream or chew some chewing gum a little bit.'
    
   \ex\label{ex:co-redup2}
   \gll Song Ailing \textbf{kan}-\textbf{le}-\textbf{kan} yupen you zou-le chulai.\\
   Song Ailing look-\textsc{pfv}-look bath.tub again walk-\textsc{pfv} out\\  \jambox{(CCL)}
   \glt `Song Ailing looked at the bath tub a little bit and walked out again.'
   \z
\z


Following the analyses above, it is clear that the reduplication failed all of the tests. 
Therefore, it seems more likely to assume the reduplication to be a morphological process rather than a syntactic one. 




%%%%%%%%%%%
%%%%%%%%%%% prev
\section{Previous analyses}\label{ch:prev}

Previous analyses on the reduplication in Mandarin Chinese and in other languages can be classified
into three groups: the reduplicant as a verbal classifier, the reduplicant as an aspect marker, and
the postulation of a special reduplication structure.


\citet{Chao1968}, \citet{Fan1964} and \citet{Xiong2016} analyzed the reduplicant in Mandarin Chinese
as a verbal classifier. 
A verbal classifier is ``a measure for verbs of action expresses the number of times an action takes place” \citep[615]{Chao1968}.
In this analysis, the first element in the reduplication is the actual verb, 
the second element is a verbal classifier borrowed from the verb, 
and \emph{yi} `one' is an optional pseudo-numeral that only has an abstract `a little bit' meaning.
Although the reduplication and the verbal classifier both serve to quantify the extent of an event and can often be used interchangeably, 
they behave differently in the following three aspects.
First, the verb and the verbal classifier can be separated, while the reduplication cannot \citep[269]{Paris2013}.
Second, unlike verbal classifiers, the \emph{yi} ‘one’ in A-\emph{yi}-A cannot be replaced by other numerals \citep[299--230]{YangWei2017}.
Third, idioms lose their idiomatic meaning when used with verbal classifiers, 
but maintain their idiomatic meaning with reduplications \citep[230--231]{YangWei2017}.
Based on these observations, it seems inappropriate to view the reduplicant as a kind of verbal classifier.


A number of studies consider the reduplicant to be a delimitative aspect marker \citep{Arcodiaetal2014, BascianoMelloni2017, YangWei2017} 
due to the delimitative meaning of the reduplication. 
\citet{Travis1999, Travis2000} also analyzed the reduplication in Tagalog as an imperfective aspect marker.
In \citet{Arcodiaetal2014} and \citegen{BascianoMelloni2017} analysis, 
the reduplication of stative and achievement verbs is structurally ruled out,
which does not fit the empirical observations we presented in Section~\ref{sec:Aktionsarten}.
The other analyses along the line of aspect marker all have problems with the A-\emph{yi}-A form, 
as the addition of \emph{yi} in the reduplication does not lead to further syntactic or semantic functions.
Moreover, although the reduplicant is postulated as a special affix that copies the phonology of the base morpheme,
the exact nature of this copying process is not formalized.

 
 \citet{Ghomeshietal2004} gave an analysis for Contrastive Reduplications (CRs) in English like (\ref{ex:cr1}) based on the Parallel Architecture proposed by \citet{Jackendoff97a, Jackendoff2002}
as shown in Figure~\ref{ghomeshi-cr}.\footnote{P = phonological unit, P/E/S CTR = prototypical/extreme/salient contrast, XP$^{min}$ = XP without its specifier}

\ea\label{ex:cr1}
I make the tuna salad, and you make the \textbf{SALAD-salad}.
\z

\begin{figure}[htbp]
\centering
%\begin{multicols}{3}
\begin{minipage}[t]{.3\linewidth}
\begin{center}
Phonology\\
%\vspace{15pt}
P$_{j, k}$ P$_k$
\end{center}
\end{minipage}
%\columnbreak
\begin{minipage}[t]{.3\linewidth}
\begin{center}
Syntax\\
\begin{forest}
for tree = {inner sep = 2pt,
	s sep = 1pt,
	anchor=children last,
    	align=center}
[X/XP$^{min}$
 [CR$^{syn}$$_j$]
 [X/XP$^{min}$$_k$]
]
\end{forest}
\end{center}
\end{minipage}
%\columnbreak
\begin{minipage}[t]{.3\linewidth}
\begin{center}
Semantics
\[
\begin{bmatrix}
Z_{k}\\
\begin{bmatrix}
P/E/S\\
CTR
\end{bmatrix}_{\!j}
\end{bmatrix}
\]
\end{center}
\end{minipage}
%\end{multicols}
\caption{Analysis for CRs in English according to \citet[344]{Ghomeshietal2004}}
\label{ghomeshi-cr}
\end{figure}
 
Applying this to the reduplication in Mandarin Chinese, the structure should be something like Figure~\ref{ghomeshi-cn}.\footnote{DELIM = delimitative}\textsuperscript{,}\footnote{
Although the reduplication in Mandarin Chinese does not have a contrastive meaning, 
we preserved the notation of CR$^{syn}$ in \citet{Ghomeshietal2004} here to simply refer to the reduplicant.
In English, it makes sense to assume CR$^{syn}$ to be a syntactic unit, because the base can be XP$^{min}$. 
But for Mandarin Chinese, the base can only be V.
As \citet[353]{Ghomeshietal2004} wrote: ``when applying to its smallest scope, X inside of a word, it has the feel of other things that attach there, i.e., morphological affixes''.
It seems that it suffices to assume the reduplication in Mandarin Chinese to be a morphological phenomenon (cf. Section~\ref{sec:word}).
We continue to call the second column ``syntax'' to preserve the consistency of the notations.
}
Further, A-\emph{le}-A can be handled as two compositional processes [[[A]-\emph{le}] -A].
Moreover, the \emph{yi} in A-\emph{yi}-A and A-\emph{le}-\emph{yi}-A can simply be viewed as a dangling phonological unit. 
In this case, the phonological unit \phonliste{ yi } is neither coindexed with a syntactic unit nor with a semantic one.

\begin{figure}[htbp]
\centering
\begin{minipage}[t]{.3\linewidth}
\begin{center}
Phonology\\
\begin{forest}
[P$_{j}$ [kan]] 
\end{forest}
\begin{forest}
[P$_{j, k}$ [kan]]
\end{forest}
\end{center}
\end{minipage}
%\columnbreak
\begin{minipage}[t]{.3\linewidth}
\begin{center}
Syntax\\
\begin{forest}
[V
 [V$_j$]
 [CR$^{syn}$$_k$]
]
\end{forest}
\end{center}
\end{minipage}
%\columnbreak
\begin{minipage}[t]{.3\linewidth}
\begin{center}
Semantics
\[
\begin{bmatrix}
\begin{bmatrix}
DELIM\\([LOOK]_{j})
\end{bmatrix}_{\!k}\\
\end{bmatrix}
\]
\end{center}
\end{minipage}
\caption{Analysis for AA following \citet{Ghomeshietal2004}}
\label{ghomeshi-cn}
\end{figure}

This analysis correctly captures the fact that the addition of \emph{yi} does not change the syntactic and semantic behavior of the reduplication.
It also provides a formal account for the phonology of the reduplication.
On the other hand, by assuming a construction specially for the reduplication, \citegen{Ghomeshietal2004} approach lost the connection between the reduplication and other aspect markers in Mandarin Chinese, unlike the affixation analysis.

 
Finally, \citet{FanSongBond2015} provided a unified HPSG analysis for the reduplication of both verbs and adjectives in Mandarin Chinese.
They considered reduplication to be a morphological process and modeled the reduplication via lexical rules.
They regarded the reduplication to function as an intensifier predicate,
which has the subtypes \type{redup\_up\_x\_rel} and \type{redup\_down\_x\_rel}.
They provided the lexical rule (\ref{avm:fsb-redup}) for reduplication in general,
and further proposed \type{redup-a-lr} and \type{redup-v-lr} as subtypes of \type{redup-type}, 
as illustrated in (\ref{avm:fsb-redup-a}) and (\ref{avm:fsb-redup-v}) respectively.
The orthography is handled separately.
The AABB form for adjectives and the ABAB form for verbs, as well as the AAB form for V-O compounds are handled as irregular derivation forms.

\ea\label{avm:fsb-redup}
\scalebox{.9}{%
\avm{
[\type*{redup-type}\\
cat|head & \1\\
val & \2\\
cont & \3 \normalfont \textsc{hook} [ltop & \4\\
			ind & \5]\\
c-cont & <[\type*{event-rel}\\
		pred & intensifier\_x\_rel\\
		lbl & \4\\
		arg1 & \5]>
]
}
$\to$
\avm{
[cat|head & \1\\
val & \2\\
cont & \3
]
}}
\z


\ea\label{avm:fsb-redup-a}
\scalebox{.9}{%
\avm{
[\type*{redup-a-lr $\subset$ redup-type}\\
cat|head & adjective\\
val & [spr <>]\\
c-cont & <[pred & redup\_up\_x\_rel]>
]
}}\\
\textsc{orthography}: A $\to$ AA; (irregular AB $\to$ AABB)
\z

\ea\label{avm:fsb-redup-v}
\scalebox{.9}{%
\avm{
[\type*{redup-v-lr $\subset$ redup-type}\\
cat|head & verb\\
cont|hook & [aspect & non-aspect]\\
c-cont & <[pred & redup\_down\_x\_rel]>
]
}}\\
\textsc{orthography}: A $\to$ AA; A $\to$ A-\emph{yi}-A; (irregular AB $\to$ ABAB)
\z

This approach provided a unified account for adjectival and verbal reduplication.
Their commonalities are captured by inheritance hierarchies of the intensifier predicates and the lexical rules.
In the case of verbal reduplication, A-\emph{yi}-A is analyzed as an alternative orthographical form of AA.
This correctly captured the intuition that AA and A-\emph{yi}-A express the same meaning and only differ from each other phonologically/orthographically.

Nevertheless, this analysis has some shortcomings.
To begin with, since the combination with aspect markers is completely forbidden, it is impossible for this approach to account for A-\emph{le}-A.
Moreover, as verbal reduplication is  considered to express a delimitative aspectual meaning,
it seems unconvincing to assume that there is no aspect information in its semantics.
We consider a semantic explanation as described in Section \ref{sec:aspM} to be more reasonable for ruling out aspect markers other than \emph{le}.
Furthermore, this account can only deal with monosyllabic reduplication and handles ABAB and AAB as irregular forms, for the reason that ABAB and AAB reduplication of AB verbs ``are not very productive in Chinese'' \citep[102]{FanSongBond2015}.
This is not true. 
\citet{BascianoMelloni2017}, \citet{MelloniBasciano2018},  \citet{Xie2020}  and \citet{Xing2000stat} all considered both AA and ABAB to be productive, 
and \citet{Xing2000stat} concluded that AAB is productive as well.
Therefore, ABAB and AAB should not be handled as  irregular forms, 
but should be derivable from lexical rules.

The shortcomings of previous analyses lead us to propose a new analysis on verbal reduplication with HPSG, that formalizes its phonology, resolves the problem of \emph{yi} and preserves the generalization on aspect marking, as we will elaborate in Section \ref{ch:HPSG-redup}. 




%%%%%%%%%%%
%%%%%%%%%%% the analysis
\section{A new HPSG analysis}\label{ch:HPSG-redup}

In what follows, we suggest a new lexical rule-based analysis of aspect marking and reduplication
using Minimal Recursion Semantics as the semantic representation formalism \citep{CFPS2005a}.

The implicational constraint in  (\ref{avm:redup}) shows the constraints on all structures of type
\emph{verbal-reduplication-lr} for Mandarin Chinese. Such structures take a verb as \textsc{lex-dtr}.  
The output reduplicates the phonology of the input verb with the possibility to have further phonological material in between.
\etag{} indicates an underspecified list which could be empty or not. A delimitative relation is
appended to the \textsc{rels} value of the input verb and it takes the event index of the input verb
as argument. The label of the output \iboxb{2} is identified with the label of the input and with the label of
the deliminitive relation, hence \type{deliminitive-rel} is treated as a modifier. Further relations can be added at the beginning of the \textsc{rels} list to allow for
the additional perfective meaning in A-\emph{le}-A and A-\emph{le}-\emph{yi}-A. 
The combination with perfective will be elaborated in the following paragraphs.

\ea
\type{verbal-reduplication-lr} \impl\\
\scalebox{.9}{%
\avm{
[
phon & \1 $\oplus$ \etag $\oplus$ \1\\
synsem & [ loc|cont [ ltop & \2 \\
                      ind  & \3 ] ]\\
rels & \etag $\oplus$ \4 $\oplus$ < [\type*{delimitative-rel}\\
                                      lbl & \2\\
 	                              arg0 & \3 ] > \smallskip\\
lex-dtr & [phon \1\\
	synsem|loc & [cat  & [ head & verb ] \\
                      cont & [ ltop & \2\\
                               ind  & \3 ]]\\
        rels \4 ] ]
}}
\label{avm:redup}
\z

To account for the variations in the phonology of the reduplication as well as the combination with the phonology and semantics of the perfective aspect marker \emph{le}, 
the type hierarchy of lexical rules in Figure~\ref{fig:typehi} is put forward. 
\begin{figure}
\centering
% \begin{tikzpicture}

% \matrix [matrix of nodes, row sep=0.7cm]
% {
% \node(redup){
% 	 \type{verbal-reduplication-lr}
% 	};
% &
% \node(le){
% 	\type{perfective-lr}
% 	};
% \\
% &
% \node(1le31){
% 	\type{perfective-reduplication-lr}
% 	};
% &
% \node(vle){
% 	\type{V-le-lr}
% 	};
% \\
% \node(AA){
% 	\type{AA-lr}
% 	};
% &
% \node(AyiA){
% 	\type{A-yi-A-lr}
% 	}; 
% &
% \node(AleyiA){
% 	\type{A-le-yi-A-lr}
% 	};
% &
% \node(AleA){
% 	\type{A-le-A-lr}
% 	};\\
% };

% \draw (redup.south) -- (AA.north);
% \draw (redup.south) -- (1le31.north);
% \draw(redup.south) -- (AyiA.north);
% \draw (le.south) -- (1le31.north);
% \draw (1le31.south) -- (AleyiA.north);
% \draw (1le31.south) -- (AleA.north);
% \draw (le.south) -- (vle.north);
% \end{tikzpicture}
\begin{forest}
type hierarchy
[,phantom
  [,phantom
    [verbal-reduplication-lr,name=verbal-reduplicaiton-lr
      [non-perfective-reduplicaiton-lr
        [a-a-lr]
        [a-yi-a-lr]]
      [,identify=!r211]]]
  [aspect-marking-lr
    [perfective-lr
      [perfective-reduplicaiton-lr%,edge to=verbal-reduplicaiton-lr
        [a-le-yi-a-lr]
        [a-le-a-lr]]
      [v-le-lr]]
    [durative-lr]
    [\ldots]]]
\end{forest}
\caption{Type hierarchy for lexical rules of verbal reduplication and \emph{le}}
\label{fig:typehi}
\end{figure}
Apart from the type \type{perfective-reduplication-lr}, which adds the inherited perfective
relation, there is a subtype \type{non-perfective-reduplication-lr}, which does not add further
relations. Hence, what is \etag in the \textsc{rels} list in (\mex{0}) is the empty list in (\mex{1}):
\ea
\type{non-perfective-verbal-reduplication-lr} \impl\\
\scalebox{.9}{%
\avm{
[
rels & \1 $\oplus$ < [ ] > \smallskip\\
lex-dtr & [ rels \1 ] ]
}
}
\z
The \textsc{rels} list of the output of the lexical rule \iboxb{1} is the \textsc{rels} list of
the daughter plus one element. Since the element is specified in the supertype it has not be
specified in (\mex{0}) again. 

\type{non-prefective-verbal-reduplication-lr} has \type{aa-lr} and \type{a-yi-a-lr} as direct subtypes.
(\ref{avm:AA}) and (\ref{avm:AyiA}) show \type{aa-lr} and \type{a-yi-a-lr}, respectively.
As subtypes of \type{ver\-bal\hyp{}re\-dup\-li\-ca\-tion\hyp{}lr} illustrated in (\ref{avm:redup}), 
both inherit the constraints on the \textsc{lex-dtr} and on the semantics of the output, 
and because of (\mex{0}) no extra material is appended to the \textsc{rels} value of the input verb
and the list containing the \type{deliminiative-rel}. In addition to the inherited constraints,
\type{aa-lr} and \type{a-yi-a-lr} specify the phonology of the output differently.
\type{aa-lr} determines that the \etag between the two phonological copies in (\ref{avm:redup}) is empty, 
whereas \type{a-yi-a-lr} specifies this list of phonological material as \phonliste{ yi }:

\ea\label{avm:constr-AA}
Constraints on lexical rules of type \type{aa-lr} and \type{a-yi-a-lr}:\\
\begin{tabular}{@{}l@{\hspace{2cm}}l@{}}
\type{aa-lr} \impl & \type{a-yi-a-lr} \impl\\
\scalebox{.9}{%
\avm{
[phon & \1 $\oplus$ \1\\
 lex-dtr & [phon & \1] ]
}} & \scalebox{.9}{%
\avm{
[phon & \1 $\oplus$ \phonliste{ yi } $\oplus$ \1\\
 lex-dtr & [phon & \1] ]
}}
\end{tabular}
\z
% \z
%
% \ea\label{avm:constr-AyiA}
% Constraints on lexical rules of type \type{a-yi-a-lr}:\\
%\hspace{2cm}
% \type{a-yi-a-lr} \impl\\ \avm{
% [phon & \1 $\oplus$ \phonliste{ yi } $\oplus$ \1\\
%  lex-dtr & [phon & \1] ]
% }
% \z

\noindent
The lexical rules with all inherited constraints are given in (\mex{1}) and (\mex{2}):

\ea\label{avm:AA}
The A-A lexical rule with all constraints inherited from the supertypes:
\scalebox{.9}{%
\avm{
[\type*{aa-lr}\\
phon & \1 $\oplus$ \1\\
synsem & [ loc|cont [ ltop & \2 \\
                      ind  & \3 ] ]\\
rels & \4 $\oplus$ < [\type*{delimitative-rel}\\
 	  		lbl & \2\\
                        arg0 & \3 ] >\\
lex-dtr & [phon & \1\\
	\punk{synsem|loc}{[cat  & [ head & verb ]\\
                           cont & [ ltop & \2\\
                                    ind  & \3 ]]}\\
    rels & \4]
]
}}
\z


\type{v-le-lr} is a direct subtype of the \type{perfective-lr}.
\type{perfective-reduplication-lr} inherits from both \type{verbal-reduplication-lr} and \type{per\-fec\-tive\hyp{}lr},
and has two subtypes \type{a-le-yi-a-lr} and \type{a-le-a-lr} itself.
\type{verbal-redupliaction-lr} is already presented in (\ref{avm:redup}). We now turn to the
constraints on \type{perfective-lr} and its subtypes.

% moved here to get space
\ea\label{avm:AyiA}
The A-yi-A lexical rule with all constraints inherited from the supertypes:
\scalebox{.9}{%
\avm{
[\type*{a-yi-a-lr}\\
phon & \1  $\oplus$ \phonliste{ yi } $\oplus$ \1\\
synsem & [ loc|cont [ ltop & \2 \\
                      ind  & \3 ] ]\\
rels &  \4 $\oplus$ < [\type*{delimitative-rel}\\
 	  		lbl & \2\\
                        arg0 & \3 ] >\\
lex-dtr & [phon & \1\\
	\punk{synsem|loc}{[cat  & [ head & verb ]\\
                           cont & [ ltop & \2\\
                                    ind  & \3 ]]}\\
    rels & \4]
]
}}
\z


\citet[246]{MuellerLipenkova2013} proposed the perfective lexical rule given in (\ref{avm:pfv-old}), adapted to the formalization adopted in the current paper.
It takes a verb as \textsc{lex-dtr}
and appends \phonliste{ le } to its phonology.
Further, it accounts for the change in semantics by appending the \textsc{rels} value of the input verb to a \type{prefective-rel}.

\ea\label{avm:pfv-old}
Perfective lexical rule adapted from \citet[246]{MuellerLipenkova2013}:\\
\scalebox{.9}{%
\avm{
[\type*{perfective-lr}\\
phon & \1 \+ \phonliste{ le }\\
\punk{synsem|cont}{ [ ltop & \2\\
                      ind & \3 ]}\\ 
rels & <[\type*{perfective-rel}\\
               lbl  & \2\\
               arg0 & \3\\
               arg1 & \4]> \+ \5\smallskip\\
lex-dtr & [phon & \1\\
	\punk{synsem|loc}{[cat & [ head & verb]\\
                           cont & [ ltop & \4\\
                                    ind & \3]]}\\
                   rels & \5]
]
}}
\z
The event variables \iboxb{3} of the input and the output verb are shared. The \textsc{ltop} of the
output of the lexical rule \iboxb{2} is the label of the perfective relation and this relation
scopes over the embedded verb. The handle of the embedded verb \iboxb{4} is the argument of the \type{perfective-rel}. 

The lexical rule suggested in (\ref{avm:pfv-old}) only explains simple perfective aspect marking with \emph{le}, where \emph{le} immediately follows the verb.
But it cannot account for the perfective aspect marking of a reduplicated verb, as \emph{le} does not occur after the reduplication, nor can \emph{le} be reduplicated together with the verb.
It can only appear between the verb and the reduplicant.
In order to accommodate \emph{le} marking for both simple and reduplicated verbs, a general perfective lexical rule as in (\ref{avm:pfv-new}) and a subtype \type{v-le-lr} as in (\ref{avm:vle}) are posited here.
Besides adding a \type{perfective-rel} in the \textsc{rels} list of the output as in (\ref{avm:pfv-old}), 
the \type{perfective-lr} in (\ref{avm:pfv-new}) allows an underspecified list to be append at the end of the \textsc{rels} list.
The \textsc{phon} value of the output makes it possible for further phonological material to occur both before and after \phonliste{ le }.

\ea\label{avm:pfv-new}
Type constraints on the type \type{perfective-lr} from which other subtypes inherit:\\
\scalebox{.9}{%
\avm{
[\type*{perfective-lr}\\
phon & \etag $\oplus$ \phonliste{ le } $\oplus$ \etag\\
\punk{synsem|cont}{ [ltop & \2\\
                     ind  & \3]}\\ 
rels & <[\type*{perfective-rel}\\
                lbl  & \2\\
                arg0 & \3\\
                arg1 & \4 ]> $\oplus$ \5 $\oplus$ \etag\smallskip\\
lex-dtr & [%phon & \1\\
	\punk{synsem|loc}{[cat & [ head & verb ]\\
                      cont & [ ltop & \4\\
                               ind  & \3]]}\\
        rels & \5]
]
}}
\z

\type{v-le-lr} as given in (\ref{avm:vle}) inherits from \type{perfective-lr} and specifies that the first element in the output \textsc{phon} list is identified with the \textsc{phon} value of the input verb
and nothing else comes after \phonliste{ le }.
Furthermore, no other list can be appended at the end of the \textsc{rels} list of the output anymore.
This corresponds to the proposal of \citet[246]{MuellerLipenkova2013} shown in (\ref{avm:pfv-old}), which accounts for the simple perfective marking of verbs.

\ea\label{avm:vle}
Structure of type \type{v-le-lr} with constraints inherited from \type{perfective-lr}:\\
\scalebox{.9}{%
\avm{
[\type*{v-le-lr}\\
phon & \1 $\oplus$ \phonliste{ le }\\
\punk{synsem|cont|ltop}{ \2}\\ 
rels & <[\type*{perfective-rel}\\
         lbl  & \2\\
         arg0 & \3\\
         arg1 & \4]> $\oplus$ \5\smallskip\\
lex-dtr & [phon & \1\\
	   cat  & [ head & verb\\
                    cont & [ltop & \4\\
                            ind & \3]]\\
           rels & \5]]
}}
\z

\type{perfective-reduplication-lr} inherits from both \type{verbal-reduplication-lr} and \type{per\-fec\-tive\hyp{}lr}.
The \textsc{phon} value of the output reduplicates the phonology of the input verb and states that there is \phonliste{ le } in between and potentially further phonological material.
The \textsc{rels} list of the output appends the \type{delimitative-rel} to the \type{perfective-rel} and the \textsc{rels} value of the input verb.
The arguments of both \type{perfective-rel} and \type{delimitative-rel} share the event index of the
input verb \iboxb{3} to ensure that they apply to the same event denoted by the input verb. The
label of the \type{deliminative-rel} and the input verb are identified (\type{deliminitive-rel} is a
modifier) and this shared label is embedded under the \type{perfective-rel}.

\ea\label{avm:pfv-redup}
Perfective and reduplication combined: type \type{perfective-reduplication-lr} with
  constraints inherited from \type{perfective-lr} and \type{verbal-reduplication-lr}:
%\resizebox{\linewidth}{!}{
\scalebox{.9}{%
\avm{
[\type*{perfective-reduplication-lr}\\
phon & \1 $\oplus$ \phonliste{ le } $\oplus$ \etag $\oplus$ \1\\
\punk{synsem|cont|ltop}{ \2}\\ 
rels & <[\type*{perfective-rel}\\
       lbl & \2\\
       arg0 & \3\\
       arg1 & \4]>
       $\oplus$ \5 $\oplus$
        <[\type*{delimitative-rel}\\
        lbl  & \4\\
	arg0 & \3]>\\
lex-dtr & [phon & \1\\
	   synsem|loc & [cat  & [ head & verb ]\\
                         cont & [ ltop & \4\\
                                  ind  & \3]]\\
           rels & \5]
]
}}
%}
\z
For example, (\ref{ex-he-tasted-the-soup-a-little-bit}) we get the following MRS representation,
where h1 and h2 correspond to the handles \ibox{2} and \ibox{4} and e1 to the event variable \ibox{3}:
\ea
h1 \sliste{ h1:perfective(e1,h2), h2:taste(e1,he,soup), h2:deliminative(e1) }
\z
So the deliminitive relation is treated as an adjunct to the main relation of the verb and the
prefective relation scopes over both the main relation and the delimitative relation.

Two subtypes of \type{perfective-reduplication-lr} are posited:
\type{a\hyp{}le\hyp{}yi\hyp{}a\hyp{}lr} %as in (\ref{avm:AleyiA}) 
and \type{a-le-a-lr} as in (\ref{avm:AleA}). %(\ref{avm:AleA}).
They take over the semantic change to the input from \type{perfective\hyp{}reduplication\hyp{}lr}, but specify the \textsc{phon} value differently.
Specifically, \type{a\hyp{}le\hyp{}yi\hyp{}a\hyp{}lr} specifies the middle phonological material as
\phonliste{ le, yi }, while \type{a-le-a} specifies it as \phonliste{ le } only.


\ea\label{avm:AleA}
\begin{tabular}{@{}l@{\hspace{2cm}}l@{}}
\scalebox{.9}{%
\avm{
[\type*{a-le-a-lr}\\
phon & \1 \+ \phonliste{ le } \+ \1\\
lex-dtr & [phon & \1
                ]
]
}}&
\scalebox{.9}{%
\avm{
[\type*{a-le-a-lr}\\
phon & \1 \+ \phonliste{ le } \+ \1\\
lex-dtr & [phon & \1
                ]
]
}}
\end{tabular}
\z


% \ea\label{avm:AleyiA}
% \avm{
% [\type*{a-le-yi-a-lr}\\
% phon & \1 \+ \phonliste{ le, yi } \+ \1\\
% lex-dtr & [phon & \1
%                 ]
% ]
% }
% \z

% \ea\label{avm:AleA}
% \avm{
% [\type*{a-le-a-lr}\\
% phon & \1 \+ \phonliste{ le } \+ \1\\
% lex-dtr & [phon & \1
%                 ]
% ]
% }
% \z


Since the above described lexical rules do not constrain the number of syllables of the input verb, but simply reduplicate its phonology  as a whole,
they can also account for the ABAB and the AB-\emph{le}-AB forms of reduplication,
as long as the input verb is disyllabic.
Notice that  the lexical rules above also produce AB-\emph{yi}-AB and AB\hyp{}\emph{le}\hyp{}\emph{yi}\hyp{}AB for disyllabic input verbs.
These forms, however, are generally considered unacceptable \citetext{\citealp[160]{BascianoMelloni2017}, \citealp[275--276]{Hong1999}, \citealp[30]{LiThompson1981}, \citealp[239]{YangWei2017}}.
On the other hand, \citet[269]{Fan1964} and \citet[143]{Sui2018} considered AB-\emph{yi}-AB and AB-\emph{le}-\emph{yi}-AB to be possible, even though they both recognized that these two forms are rare.
Indeed, a few examples of AB-\emph{yi}-AB and AB-\emph{le}-\emph{yi}-AB were found (\ref{ex:ABleyiAB}).

%\footnotetext[11]{\emph{Yuanqu xuan: Luzhailang [Selected Yuanqu: Luzhailang]}, as cited in \citet[15]{Zhang2000}}
%\footnotetext[12]{\emph{Yuan Ming juan: Piaotongshi [Yuan and Ming volume: Piaotongshi]}, 308, as cited in \citet[15]{Zhang2000}}
\footnotetext[11]{Rou, Shi. 1975. \emph{Roushi xiaoshuo xuanji [Selected novels of Roushi]}, 31. Beijing: People's Literature Publishing House.}
\footnotetext[12]{Li, Jieren. 1962. \emph{Da bo [Great wave]}, 3rd band, 171. Beijing: The Writers Publishing House.}
\ea\label{ex:ABleyiAB}
 %\ea\label{ex:AByiAB-lu}
 %\gll ni yu wo \textbf{zhengli-yi-zhengli}.\footnotemark\\
 %you let me arrange-one-arrange\\
 %\glt `Let me arrange it a little bit!''
 
 %\ex\label{ex:AByiAB-pu}
 %\gll ni \textbf{dating-yi-dating}.\footnotemark\\
 %you inquire-one-inquire\\
 %\glt `Inquire about it a little bit!''
 
 \ea\label{ex:AByiAB-rou2}
 \gll ta \textbf{weixiao-le-yi-weixiao}, you \textbf{mingxiang-le-yi-mingxiang}.\footnotemark\\
 he smile-\textsc{pfv}-one-smile and meditate-\textsc{pfv}-one-meditate\\
 \glt `He smiled a little bit and meditated a little bit.'
 
 \ex\label{ex:AByiAB-li}
 \gll feichang yansu de ba jinshi yanjing \textbf{duanzheng-le-yi-duanzheng}.\footnotemark\\
 very seriously \textsc{de} \textsc{ba} nearsighted glasses straighten-\textsc{pfv}-one-straighten\\
 \glt `Very seriously put the nearsighted glasses straight quickly.'
 
 \ex\label{ex:ABleyiAB-ccl}
 \gll jiduo sanluan-zhe de chuan li de dengguang, ye huyinhumie de \textbf{bianhuan-le-yi-bianhuan} weizhi.\\
  many scattered-\textsc{dur} \textsc{de} boat in \textsc{de} light also flicker \textsc{de} change-\textsc{pfv}-one-chang position\\ \jambox{(CCL)}
 \glt `Many scattered lights in the boats also changed their positions a little bit, flickering.'
 \z
\z

This suggests that  even though AB-\emph{yi}-AB and AB-\emph{le}-\emph{yi}-AB might be degraded, they are not ungrammatical \emph{per se}.
The reason for this degradedness is probably phonological, namely  AB-\emph{yi}-AB and AB-\emph{le}-\emph{yi}-AB contain too many syllables \citetext{\citealp[274]{Fan1964}, \citealp[143]{Sui2018}, \citealp[239]{YangWei2017}, \citealp[15]{Zhang2000}}.
We argue that  the reduced acceptability of AB-\emph{yi}-AB and AB\hyp{}\emph{le}\hyp{}\emph{yi}\hyp{}AB is due to their phonological length and not their grammaticality.
Thus, they can still be produced via the lexical rules posited above, but are ruled out or degraded due to a general phonological constraint.


AAB, A-\emph{yi}-AB, A-\emph{le}-AB, AA-\emph{kan} and A-\emph{kan}-\emph{kan} can also be accounted for by the lexical rules proposed in this section.
They can be analyzed as compounds out of a reduplicated monosyllabic verb and another element.
Specifically, AAB, A-\emph{yi}-AB and A-\emph{le}-AB can be considered as the compound of a reduplicated monosyllabic verb (A) and a noun (B).\footnote{
\citet{Huang1984} and \citet{Her1996, Her2010} argued that some of this kind of structures are compounds, some are phrases, and some have dual status (both compounds and phrases).
Following this approach, AAB, A-\emph{yi}-AB and A-\emph{le}-AB can (also) be considered as the phrasal combination of a reduplicated verb and its object.
}
AA-\emph{kan} can be regarded as the compound of a reduplicated monosyllabic verb (A) and the verb \emph{kan} `look',
whereas A-\emph{kan}-\emph{kan} is the compound of a monosyllabic verb (A) and the reduplication of \emph{kan} `look'.
A-\emph{yi}-A-\emph{kan} is also possible, although rare, presumably due to its length as well.
An inquiry in CCL found 55 hits of A-\emph{yi}-A-\emph{kan}.
A sample is listed in (\ref{ex:AyiAkan}).

\ea\label{ex:AyiAkan}
  \ea
  \gll danshi dui fa mei fa-guo hege-zheng, yijing shuo bu qing le, xuyao \textbf{cha-yi-cha-kan}.\\
  but about issue not issue-\textsc{exp} conformity-certificate already say not clealy \textsc{ptc} need check-one-chek-look\\ \jambox{(CCL)}
  \glt `But one already cannot say it clearly anymore, whether a certificate of conformity is issued or not. One needs to have a check and see.'
  
  \ex
  \gll da-laoban-men yao \textbf{deng-yi-deng-kan}\\
  big-boss-\textsc{pl} need wait-one-wait-look\\ \jambox{(CCL)}
  \glt `Big bosses need to wait a little bit and see.'
  
  \ex
  \gll furen ni dao \textbf{shu-yi-shu-kan}, zhe zhu hua de huaduo gong you ji zhong yanse.\\
  madam you just count-one-count-look this \textsc{clf} flower \textsc{de} blossom in.total have how.many \textsc{clf} color\\ \jambox{(CCL)}
  \glt `Madam, just try to count and see how many color does the blossom of this flower have in total.'
  \z
\z
Due to the prominent tentative, trying meaning of AA-\emph{kan} and A-\emph{kan}-\emph{kan}, they are not compatible with the perfective aspect marker \emph{le} semantically,
as one usually cannot try something that is already realized.
Unacceptable structures such as  A\hyp{}\emph{le}\hyp{}A\hyp{}\emph{kan} and A\hyp{}\emph{kan\hyp{}le\hyp{}kan} are thus semantically ruled out.

The current analysis provides a unified account for all forms of delimitative verbal reduplication in Mandarin Chinese.
Like in \citet{FanSongBond2015}, \emph{yi} is handled as a phonological element which does not make any contribution to the semantics,
 and an inheritance hierarchy is used to capture the commonalities among different forms of reduplication.
But the present proposal also reflects the connection between the reduplication and aspect marking via multiple inheritance.
This account makes use of a semantic mechanism, which correctly rules out the aspect marking other than \emph{le}.
By providing a semantic explanation, this mechanism seems less \emph{ad hoc} than the one used in \citet{FanSongBond2015}, which simply assumed that the reduplication cannot combine with aspect information.
The present approach also has a broader coverage of the forms of verbal reduplication than \citet{FanSongBond2015}.
Furthermore, all the forms are derivable from the lexical rules proposed here, so that there is no need to resort to irregular lexicon entries, and the productivity of these forms are correctly captured.
In sum, the analysis proposed in this paper seems to possess greater explanatory power and resolves the problems of previous studies.


%%%%%%%%%%%
%%%%%%%%%%% sum
\section{Conclusion}

The current study provides an HPSG account for verbal reduplication in Mandarin Chinese.
We presented empirical evidence that the reduplication is possible with all \emph{Aktionsarten}.
We gave a semantic explanation for the incompatibility of the reduplication with aspect markers other than \emph{le}.
We  argued that the reduplication is a morphological rather than a syntactic process.
We modeled the reduplication as a lexical rule
and the different forms of reduplication are captured in an inheritance hierarchy using underspecified lists.
The interaction between verbal reduplication and aspect marking is handled by multiple inheritance.
This analysis is compatible with both mono- and disyllabic verbs, 
so that all productive forms of reduplication are derivable by lexical rules.

The analysis is implemented as part of the CoreGram project \citep{MuellerCoreGram} in a Chinese
grammar in the TRALE system \citep*{MPR2002a-u,Penn2004a-u}.


\renewrobustcmd{\disambiguate}[3]{#1}
%\bibliographystyle{unified} 
%\bibliography{Bib}

{\sloppy
\printbibliography[heading=subbibliography,notkeyword=this] 
}


\end{document}


%      <!-- Local IspellDict: en_US -->
